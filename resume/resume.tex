% FortySecondsCV LaTeX template
% Copyright © 2019-2020 René Wirnata <rene.wirnata@pandascience.net>
% Licensed under the 3-Clause BSD License. See LICENSE file for details.
%
% Please visit https://github.com/PandaScience/FortySecondsCV for the most
% recent version! For bugs or feature requests, please open a new issue on
% github.
%
% Contributors
% ------------
% * ifokkema
% * Bertbk
% * Hespe
% * esben
%
% Attributions
% ------------
% * fortysecondscv is based on the twentysecondcv class by Carmine Spagnuolo
%   (cspagnuolo@unisa.it), released under the MIT license and available under
%   https://github.com/spagnuolocarmine/TwentySecondsCurriculumVitae-LaTex
% * further attributions are indicated immediately before corresponding code


%-------------------------------------------------------------------------------
%                             ADDITIONAL PACKAGES
%-------------------------------------------------------------------------------
\documentclass[
	a4paper,
	% showframes,
	% vline=2.2em,
	maincolor=cvartifakt,
	sidecolor=cvgray,
	sidebartextcolor=cvwhite,
	sectioncolor=cvartifakt,
	subsectioncolor=cvgray,
	% itemtextcolor=cvgray,
	sidebarwidth=0.36\paperwidth,
    topbottommargin=0.04\paperheight,
	% leftrightmargin=20pt,
	% profilepicsize=4.5cm,
	% profilepicborderwidth=3.5pt,
	% profilepicstyle=profilecircle,
	% profilepiczoom=1.0,
	% profilepicxshift=0mm,
	% profilepicyshift=0mm,
	% profilepicrounding=1.0cm,
	% logowidth=4.5cm,
	% logospace=5mm,
	% logoposition=before,
]{fortysecondscv}

% improve word spacing and hyphenation
\usepackage{microtype}
\usepackage{ragged2e}
\usepackage{enumitem}

% uncomment in case you don't want any hyphenation
% \usepackage[none]{hyphenat}

% take care of proper font encoding
\ifxetexorluatex
	\usepackage{fontspec}
	\defaultfontfeatures{Ligatures=TeX}
%	\newfontfamily\headingfont[Path = fonts/]{segoeuib.ttf} % local font
\else
	\usepackage[utf8]{inputenc}
	\usepackage[T1]{fontenc}
%	\usepackage[sfdefault]{noto} % use noto google font
\fi

% enable mathematical syntax for some symbols like \varnothing
\usepackage{amssymb}

% bubble diagram configuration
\usepackage{smartdiagram}
\smartdiagramset{
	% default font size is \large, so adjust to harmonize with sidebar layout
	bubble center node font = \footnotesize,
	bubble node font = \footnotesize,
	% default: 4cm/2.5cm; make minimum diameter relative to sidebar size
	bubble center node size = 0.4\sidebartextwidth,
	bubble node size = 0.25\sidebartextwidth,
	distance center/other bubbles = 1.5em,
	% set center bubble color
	bubble center node color = maincolor!70,
	% define the list of colors usable in the diagram
	set color list = {maincolor!10, maincolor!40,
	maincolor!20, maincolor!60, maincolor!35},
	% sets the opacity at which the bubbles are shown
	bubble fill opacity = 0.8,
}


%-------------------------------------------------------------------------------
%                            PERSONAL INFORMATION
%-------------------------------------------------------------------------------
%% mandatory information
% your name
\cvname{Thibault Hazelart}
% job title/career
\cvjobtitle{Site Reliability Engineer}

%% optional information
% profile picture
\cvprofilepic{pics/profile.png}

% NOTE: ordering in sidebar will mimic the following order
% date of birth
% \cvbirthday{31 mars 1992}
% short address/location, use \newline if more than 1 line is required
\cvaddress{Toulouse area, FRANCE}
% phone number
% \cvphone{+33 6 31 45 99 52}
% email address
\cvmail{thibault.hazelart@gmail.com}
% any other custom entry
\cvcustomdata{\faLinkedin}{\href{https://linkedin.com/in/thazelart}{LinkedIn}}
\cvcustomdata{\faGithub}{\href{https://github.com/thazelart}{GitHub}}



%-------------------------------------------------------------------------------
%                              SIDEBAR 1st PAGE
%-------------------------------------------------------------------------------
% add more profile sections to sidebar on first page
\addtofrontsidebar{
    \color{sidebartextcolor}
	
	\profilesection{A propos de moi}
	\aboutme{
	    Automation fan from the beginning, I would like to bring my skills and spirit to a company that trust as much in the open source than I do.
	}
	
	\profilesection{Compétences}
		\skill{\faCloud}{Cloud}
			\pointskill[1em]{}{AWS}{4}
			\pointskill[1em]{}{GCP}{4}
			\pointskill[1em]{}{Azure}{3}
		\skill{\faCog}{Automation}
		    \pointskill[1em]{}{Terraform}{4}
		    \pointskill[1em]{}{Ansible}{4}
		\skill{\faCode}{Programming}
		    \pointskill[1em]{}{Bash}{4.5}
		    \pointskill[1em]{}{Golang}{4}
		    \pointskill[1em]{}{Python}{3.5}
		\skill{\faShip}{Misc}
			\pointskill[1em]{}{Observability}{4}
		    \pointskill[1em]{}{Kubernetes}{4}
		    \pointskill[1em]{}{Kafka}{3}
		
	% include gosquare national flags from https://github.com/gosquared/flags;
	% naming according to ISO 3166-1 alpha-2 country codes
	\graphicspath{{pics/flags/}}
	\profilesection{Langues}
	    \pointskill{\flag{GB.png}}{English}{4}
		\pointskill{\flag{FR.png}}{French}{5}
}


%-------------------------------------------------------------------------------
%                              PERSONNAL TRICKS
%-------------------------------------------------------------------------------
% \setitemize[0]{label=$\rightarrow$, leftmargin=*, noitemsep}
\setlist[itemize]{label=$\rightarrow$, leftmargin=*, noitemsep, topsep=0pt, parsep=0pt, nosep}


\makeatletter
\newcommand\thefontsize{\f@size pt}
\makeatother
%-------------------------------------------------------------------------------
%                         TABLE ENTRIES RIGHT COLUMN
%-------------------------------------------------------------------------------
\begin{document}

\makefrontsidebar{}

\cvsection{\faSuitcase\hspace{5pt} Working experience}
\begin{cvtable}[2]
% \cvitem{\thefontsize}{\thefontsize}{\thefontsize}
%	    {\thefontsize}
	\cvitem{oct 2022}{Site Reliability Engineer}{Artifakt, Remote}
	{
		\begin{itemize}
			\item Creation of a \textbf{Golang / gRPC API} to abstract all the cloud technologies: kubernetes, argo-workflow, git, kaniko.
			\item \textbf{Design} and \textbf{implementation} of the new product cloud architecture: GCP, Terraform, ArgoCD, Helm, ExternalDNS, ExternalSecrets, CertManager, Traefik.
			\item Design of the new product \textbf{observability}: Prometheus, Grafana, Loki.
			\item Writing of technical documentation.
		\end{itemize}
	}
	\cvitem{2021 - 2022}{Site Reliability Engineer}{Continental, Toulouse}
	{
		\textcolor{maincolor}{2022 | Key as a Service}
			\begin{itemize}
				\item \textbf{Leader of the SRE team of 8 people} (Operations and platform).
				\item Technical lead: Onboarding / support / training of new team members, code reviews.
				\item Organisation of the team: Setup the \textbf{Agile} organisation and rituels, \textbf{prioritisation}.
				\item Reporting: Proposals and implementation of processes for the continuous improvement of the team both from an organisational and technical point of view.
			\end{itemize}
			\vspace{8pt}
			\textcolor{maincolor}{2021 | Key as a Service}
			\begin{itemize}
				\item Application maintenance: Kubernetes, Hashicorp Vault, Envoy, \textbf{AWS}, MongoDB, PostgresQL, Golang.
				\item Automation of infrastructure and application installation: Terraform, Helm, Gitlab, Bash, Jenkins.
				\item Continuous improvement of monitoring and alerting: \textbf{Prometheus}, \textbf{AlertManager}, \textbf{Grafana}, Kibana.
				\item \textbf{Recruitment} for the team in France and India.
			\end{itemize}
	}
	\cvitem{2019 - 2020}{DevOps Engineer / SRE}{Conserto, Toulouse}
	{
		\textcolor{maincolor}{2020 | La Poste}
		\begin{itemize}
			\item \textbf{Technical leader} and architecture of the automation tools: ansible, ansible tower, terraform.
			\item Implementation of \textbf{CI} and \textbf{development best practices}: Git, Ansible, \textbf{Terraform}, bash, packer.
			\item Implementation of \textbf{agility}.
		\end{itemize}
	}
	\cvitem{2015 - 2019}{DevOps Engineer / SRE}{Decathlon, Lille}
	{
		\begin{itemize}
			\item Architecture of cloud projects: terraform, \textbf{GCP}, \textbf{Kubernetes}.
			\item Sharing the SRE culture with application teams.
			\item Automation of deployments and recurring tasks with \textbf{ansible}, \textbf{python}, \textbf{bash} and \textbf{Rundeck}.
			\item Run platforms on \textbf{CentOS/RedHat} from integration to production.
			\item System \textbf{trainer} and support of Chinese teams.
		\end{itemize}
	}
\end{cvtable}

\cvsection{\faGraduationCap\hspace{5pt} Education}
\begin{cvtable}[2]
	\cvitem{2020}{Microsoft Azure Administrator (AZ-104)}{M2i Montpellier}
		{4 days of training dedicated to Microsoft Azure administration.}
	\cvitem{2019}{Architecting with GCP: Infrastructure}{SFEIR institute}
		{3 days of training dedicated to the architecture with GCP.}
	\cvitem{2018}{Kafka: Confluent Operations}{Zenika Academy}
		{3 days of training dedicated to the architecture and administration of a Kafka cluster.}
	\cvitem{2013 - 2015}{Master TIIR (mention bien)}{Université Lille 1}
		{Master degree dedicated to system, network and security.}
\end{cvtable}

\vspace{-11pt}
\end{document}